\documentclass[12pt,a4paper]{article}
\usepackage[utf8]{inputenc}
\usepackage{amsmath}
\usepackage{amsfonts}
\usepackage{amssymb}
\author{Angel Diaz Cervantes Amieva}
\title{Arquitectura Movil}
\begin{document}
\paragraph{¿Qué es la arquitectura móvil?}
\paragraph{}

La arquitectura de aplicaciones móviles se refiere a aquellos sistemas que permiten construir y estructurar los recursos de diseño que forman parte de la composición de la app.
Describe los patrones y las técnicas que se utilizan para diseñar y desarrollar aplicaciones. 
\\
\\La arquitectura le proporciona un plan y las prácticas recomendadas que debe seguir para diseñar una aplicación bien estructurada.
\\
En una arquitectura de aplicaciones, habrá servicios de frontend y de backend.El desarrollo de frontend se refiere a la experiencia del usuario con la aplicación, mientras que el de backend implica proporcionar acceso a los datos, los servicios y otros sistemas que permiten el funcionamiento de la aplicación.
\paragraph{Tipos de arquitectura de aplicaciones}
\paragraph{.}
Arquitectura cliente-servidor
\\
Esta es la forma más común de arquitectura de aplicaciones web.
\\
Consta de dos partes principales: una del lado del cliente y otra del lado del servidor.
\\
La parte cliente incluye cualquier interfaz de usuario o código front-end, mientras que la parte servidor se encarga del back-end, es decir el almacenamiento de datos, la lógica de negocio y la comunicación con servicios externos como procesadores de pagos o bases de datos.
\paragraph{.}
Arquitectura de una sola página o Single Page Application
\\
La arquitectura Single-Page Application (SPA) es un enfoque popular para crear aplicaciones web.
\\
Con la arquitectura SPA, todo el código necesario para ejecutar la aplicación se almacena en una página a la que se puede acceder desde cualquier dispositivo conectado a Internet.
\\
Este tipo de arquitectura utiliza JavaScript y HTML5 para crear interfaces de usuario dinámicas, rápidas y con capacidad de respuesta.
\paragraph{.}
Arquitectura de Aplicaciones Web Progresivas
\\
La arquitectura de aplicaciones web progresivas (PWA) es un enfoque del desarrollo de aplicaciones web que combina lo mejor de las aplicaciones nativas y de las aplicaciones web tradicionales.
\\
Proporciona una experiencia de usuario similar a la de una app nativa, pero con toda la flexibilidad de una aplicación web
\\
Las PWA utilizan service workers, que se trata de código escrito en JavaScript para habilitar la funcionalidad offline y las notificaciones de audio.
\paragraph{.}
Arquitectura de Renderizado del Lado del Servidor
\\
Las aplicaciones web eficaces y fáciles de usar suelen basarse en una arquitectura de renderizado del lado del servidor (SSR).
\\
Con este enfoque, el servidor renderiza HTML y CSS mientras el navegador del cliente se encarga del resto de tareas. Los usuarios disfrutan así de una experiencia más rápida, ya que no tienen que esperar a que se descargue o analice JavaScript.
\paragraph{.}
Arquitectura de Microservicios
\\
Las aplicaciones web eficientes y escalables pueden construirse utilizando una arquitectura de microservicios.
\\
En este enfoque, la aplicación se divide en componentes o servicios más pequeños que se gestionan de forma independiente.
\\
Esto permite a los desarrolladores crear una aplicación más rápidamente, ya que elimina la necesidad de reescribir código para los distintos componentes.
\end{document}
\ref{https://www.dongee.com/tutoriales/cual-es-la-arquitectura-de-una-aplicacion-web/}
\ref{https://www.redhat.com/es/topics/cloud-native-apps/what-is-an-application-architecture}