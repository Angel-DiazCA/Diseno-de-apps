\documentclass[12pt,a4paper]{article}
\usepackage[utf8]{inputenc}
\usepackage{amsmath}
\usepackage{amsfonts}
\usepackage{amssymb}
\author{Angel Diaz Cervantes Amieva}
\title{Patrones de diseño para aplicaciones moviles}
\begin{document}
\textbf{Introducción}\\
Los patrones de diseño son simples soluciones para los problemas comunes que se presentan al hacer o al desarrollar un proyecto de software, éstos nos ayudan guiándonos con consejos o sugerencias de cómo aplicarlos y permitiéndonos re-utilizarlos.\\
Los patrones de diseño se clasifican de acuerdo al tipo de enfoque necesario.\\
Los patrones idóneos para la creación de aplicaciones móviles según su clasificación son los siguientes:
\paragraph{•Patrones creacionales}
\paragraph{}
Son los que facilitan la tarea de creación de nuevos objetos, de tal forma que el proceso de creación pueda ser desacoplado de la implementación del resto del sistema. En este grupo están los patrones que sirven de guía para construir objetos. Además, independizan el ¿Qué?, ¿Quién?, ¿Cómo? y ¿Cuándo? en la creación del objeto.
\paragraph{•Patrones estructurales}
\paragraph{}
Un patrón estructural de objetos describe la forma en que se componen objetos para obtener nueva funcionalidad, además se añade la flexibilidad de cambiar la composición en tiempo de ejecución”. En este grupo encontramos los patrones que organizan las clases y objetos de características similares y así formar estructuras más grandes.
\end{document}