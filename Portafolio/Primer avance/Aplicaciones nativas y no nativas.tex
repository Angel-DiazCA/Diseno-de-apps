\documentclass[12pt,a4paper]{article}
\usepackage[utf8]{inputenc}
\usepackage{amsmath}
\usepackage{amsfonts}
\usepackage{amssymb}
\author{Angel Diaz Cervantes Amieva}
\title{Aplicaciones nativas y no nativas}
\begin{document}
\paragraph{Aplicaciones nativas}
\paragraph{}
Las aplicaciones nativas son aquellas que se desarrollan para un sistema operativo específico, principalmente Android o iOS ya que son los más conocidos y utilizados en los dispositivos móviles mundialmente.
\\
\\
Se llaman aplicaciones nativas debido a que se desarrollan para el sistema operativo nativo de cada dispositivo. Este tipo de aplicaciones móviles son aquellas que nos descargamos en las tiendas de apps como pueden ser Play Store (Android) y App Store (iOS).
\\
\\
Para una app nativa Android el lenguaje que se debe utilizar es Java mientras que para hacer una app nativa iOS los lenguajes de programación utilizados son Objective-C y Swift.
\paragraph{Ventajas de las aplicaciones nativas}
\paragraph{.}
Gran nivel de personalización
\paragraph{.}
Uso sin conexión a internet
\paragraph{.}
Seguridad
\paragraph{Desventajas de las aplicaciones nativas}
\paragraph{.}
Mayor costo
\paragraph{.}
Código único
\paragraph{.}
Más tiempo para desarrollar
\paragraph{.}
Equipo especializado
\paragraph{}
\paragraph{}
\paragraph{}
\paragraph{Aplicaciones no nativas}
\paragraph{}
Consisten en un solo desarrollo que se puede desplegar en varios sistemas operativos y también son fácilmente accesibles a través de un navegador.\\
Una tendencia de mejora en las aplicaciones no nativas las ha convertido en la opción preferible para realizar todo tipo de soluciones para el cliente final. Son más rápidas y económicas de desarrollar, más fáciles de sacar al mercado, y más escalables.
\paragraph{Ventajas de las aplicaciones no nativas}
\paragraph{.}
Ahorro en los costes de producción y mantenimiento
\paragraph{.}
Desarrollos escallables
\paragraph{.}
Mayor eficiencia de los procesos
\paragraph{.}
Accesibiladad
\paragraph{Desventajas de las aplicaciones no nativas}
\paragraph{.}
Gráficos mas bajos
\paragraph{.}
Mayor número de fallas durante la ejecución
\paragraph{.}
Experiencia de usuario deficiente
\end{document}